%%%%%%%%%%%%%%%%%
%
%%%%%%%%%%%%%%%%

\documentclass[10pt,a4paper,ragged2e,withhyper]{altacv}

% Change the page layout if you need to
\geometry{left=1.25cm,right=1.25cm,top=1.5cm,bottom=1.5cm,columnsep=1.2cm}

% The paracol package lets you typeset columns of text in parallel
\usepackage{paracol}
\usepackage{fontspec}
\usepackage{roboto}
\usepackage{lato}
% If using xelatex or lualatex:
\setmainfont{Roboto Slab}
\setsansfont{Lato}
\renewcommand{\familydefault}{\sfdefault}

% Change the colours if you want to
\definecolor{SlateGrey}{HTML}{2E2E2E}
\definecolor{LightGrey}{HTML}{666666}
\definecolor{DarkPastelRed}{HTML}{450808}
\definecolor{PastelRed}{HTML}{8F0D0D}
\definecolor{GoldenEarth}{HTML}{E7D192}
\colorlet{name}{black}
\colorlet{tagline}{PastelRed}
\colorlet{heading}{DarkPastelRed}
\colorlet{headingrule}{GoldenEarth}
\colorlet{subheading}{PastelRed}
\colorlet{accent}{PastelRed}
\colorlet{emphasis}{SlateGrey}
\colorlet{body}{LightGrey}


% Change some fonts, if necessary
\renewcommand{\namefont}{\Huge\rmfamily\bfseries}
\renewcommand{\personalinfofont}{\footnotesize}
\renewcommand{\cvsectionfont}{\LARGE\rmfamily\bfseries}
\renewcommand{\cvsubsectionfont}{\large\bfseries}


% Change the bullets for itemize and rating marker
% for \cvskill if you want to
\renewcommand{\itemmarker}{{\small\textbullet}}
\renewcommand{\ratingmarker}{\faCircle}

\begin{document}
\name{Daniel Rush}
\tagline{Senior Software Engineer}
%% \photoL{2.5cm}{Yacht_High,Suitcase_High}

\personalinfo{%
  \email{rush.daniel95@gmail.com}
  \location{San Francisco, California}
  \linkedin{daniel-rush95}
  \github{theDrsh}
}

\makecvheader
%% Depending on your tastes, you may want to make fonts of itemize environments slightly smaller
% \AtBeginEnvironment{itemize}{\small}

%% Set the left/right column width ratio to 6:4.
\columnratio{0.55}

% Start a 2-column paracol. Both the left and right columns will automatically
% break across pages if things get too long.
\begin{paracol}{2}
\cvsection{Experience}

\cvevent{Senior Software/Firmware Engineer}{\href{https://carbon3d.com}{Carbon, Inc.}}{June 2017 - Current}{Redwood City, California}
\begin{itemize}
\item Main firmware engineer on the \underline{\textbf{\href{https://www.carbon3d.com/products/m3-3d-printer/}{M3 and M3 Max}}} project. Developed project from conception through engineering build, alpha, beta, and production
\item Maintained firmware for all Carbon products and subsystems
\item Developed automation, code generation, hardware-in-the-loop tests, and tools for building and testing firmware and hardware
\item Main Owner of core subsystems that enable printing, including the Oxygen control, and motion control subsystems
\end{itemize}

\divider

\cvevent{Associate Firmware Engineer}{\href{https://www.mechatronicscenter.com/}{California Mechatronics Center}}{November 2016 - May 2017 }{Chico, California}
\begin{itemize}
\item Developed firmware for an earthquake sensing system using MQTT and USB device for data transmission and redundant storage in case of blackout or loss of internet
\item Worked on Verilog HDL for FPGA based sensor system with mass parallel sampling and data transfer
\item Developed encoding and decoding of data from FPGA to microcontroller
\item Overcame hardware constraints with custom serialization library for data transmission with limited RAM
\end{itemize}

\cvsection{Languages}
\cvskill{C++}{5}
\cvskill{C}{5}
\cvskill{Python}{4}
\cvskill{Verilog}{2}
\cvskill{Bash}{1}

\divider

\cvsection{Proficiencies}

\cvtag{I2C}
\cvtag{SPI}
\cvtag{UART}
\cvtag{USB}
\cvtag{MQTT}
\cvtag{Protobuf}
\cvtag{Docker}
\cvtag{System Integration}
\cvtag{Board Bring-up}
\cvtag{Linux}
\cvtag{Bazel}
\cvtag{Cmake}
\cvtag{Jenkins}
\cvtag{HIL Testing}
\cvtag{RTOS}
\cvtag{YAML}
\cvtag{Mako}


\cvsection{Strengths}

\cvtag{Detail Oriented}
\cvtag{Self-Starter}
\cvtag{Dedicated Teammate}
\cvtag{Communicator}
\cvtag{Persistent Problem-Solver}


%% Switch to the right column. This will now automatically move to the second
%% page if the content is too long.
\switchcolumn

\cvsection{Education}

\cvevent{B.Sc.\ Mechatronics Engineering}{California State University, Chico}{Aug 2013 - May 2018}{}

\divider

\cvevent{B.Sc.\ Computer Engineering}{California State University, Chico}{Aug 2013 - May 2018}{}

\divider

\cvsection{Projects}

\cvevent{M3 And M3 Max}{\href{https://carbon3d.com}{Carbon}}{}{}
\begin{itemize}
\item Worked with hardware engineers, product managers, and print development engineers to create firmware requirements.
\item Built all new subsystems in house to create Carbon's "smartest" printer
\item Wrote firmware with extensibilty, ease of debug, and robustness in mind to create a lasting product.
\item Brought up and integrated 3 systems, from the first engineering build to the first production unit.
\item Dealt with many challenges including supply chain shortages which necessitated new drivers from scratch on a short timeline.
\end{itemize}

\divider

\cvevent{Motor Controller}{\href{https://carbon3d.com}{Carbon}}{}{}
\begin{itemize}
\item Developed from conception a FOC Motor controller, with cascaded control loop allows for torque, velocity, relative position, or absolute position control.
\item Created drivers for USB Logging, UART API interface, SPI Controlled FET-Driver chip, I2C EEPROM hardware version control.
\item Designed and implemented hardware fault detection and safety states to prevent damage to mechanics and electronics.
\item Architected HIL(Hardware-In-the-Loop) tester using dynamometer for system and firmware validation.
\end{itemize}

\end{paracol}
\newpage

\cvsection{Other Projects}

\cvevent{Open Source Firmware communications}{\href{https://github.com/theDrsh/uComs}{uComs}}{}{}
\begin{itemize}
\item Ongoing project to make starting an embedded project's communication protocol easy and generated
\item Integrates all the low-level intricacies of communicating between a host and a device, providing a boilerplate for communications
\end{itemize}

\divider

\cvevent{Microtest}{\href{https://carbon3d.com}{Carbon}}{}{}
\begin{itemize}
\item Developed a firmware image which was able to test core interfaces of the microcontroller
item Server-side application to interface the microcontroller with gtest on Linux, using  \underline{\textbf{\href{https://github.com/nanopb/nanopb}{Nanopb}}}
\item Developed Jenkins-based job to automatically run tests upon pull request.
\item Created ability write firmware that could be automatically called and tested via Jenkins.
\item Recorded statistics like execution time and logs on target processor.
\end{itemize}

\divider

\cvevent{Code Generation Projects}{\href{https://carbon3d.com}{Carbon}}{}{}
\begin{itemize}
\item Worked to develop python-generated firmware to automate different parts of the firmware stack:
	\begin{itemize}
	\item Server <-> embedded system communication.
	\item microcontroller <-> microcontroller communication and commands.
	\item Hardware faults and alerts generated through entire software stack micro -> server -> cloud.
	\end{itemize}
\end{itemize}

\divider

\cvevent{Board Bootloader and Over-The-Air(OTA) Update Software}{\href{https://carbon3d.com}{Carbon}}{}{}
\begin{itemize}
\item Helped build the bootloaders for smart subsystems of printer
\item Extended OTA updates to incorporate updating all smart subsystems.
\end{itemize}

\divider

\end{document}
