%%%%%%%%%%%%%%%%%
%
%%%%%%%%%%%%%%%%

\documentclass[10pt,a4paper,ragged2e,withhyper]{altacv}

% Change the page layout if you need to
\geometry{left=1.25cm,right=1.25cm,top=1.5cm,bottom=1.5cm,columnsep=1.2cm}

% The paracol package lets you typeset columns of text in parallel
\usepackage{paracol}
\usepackage{fontspec}
\usepackage{roboto}
\usepackage{lato}
% If using xelatex or lualatex:
\setmainfont{Roboto Slab}
\setsansfont{Lato}
\renewcommand{\familydefault}{\sfdefault}

% Change the colours if you want to
\definecolor{SlateGrey}{HTML}{2E2E2E}
\definecolor{LightGrey}{HTML}{666666}
\definecolor{DarkPastelRed}{HTML}{450808}
\definecolor{PastelRed}{HTML}{8F0D0D}
\definecolor{GoldenEarth}{HTML}{E7D192}
\colorlet{name}{black}
\colorlet{tagline}{PastelRed}
\colorlet{heading}{DarkPastelRed}
\colorlet{headingrule}{GoldenEarth}
\colorlet{subheading}{PastelRed}
\colorlet{accent}{PastelRed}
\colorlet{emphasis}{SlateGrey}
\colorlet{body}{LightGrey}


% Change some fonts, if necessary
\renewcommand{\namefont}{\Huge\rmfamily\bfseries}
\renewcommand{\personalinfofont}{\footnotesize}
\renewcommand{\cvsectionfont}{\LARGE\rmfamily\bfseries}
\renewcommand{\cvsubsectionfont}{\large\bfseries}


% Change the bullets for itemize and rating marker
% for \cvskill if you want to
\renewcommand{\itemmarker}{{\small\textbullet}}
\renewcommand{\ratingmarker}{\faCircle}

\begin{document}
\name{Daniel Rush}
\tagline{Senior Software Engineer}
%% \photoL{2.5cm}{Yacht_High,Suitcase_High}

\personalinfo{%
  \email{rush.daniel95@gmail.com}
  \location{San Francisco, California}
  \linkedin{daniel-rush95}
  \github{theDrsh}
}

\makecvheader
%% Depending on your tastes, you may want to make fonts of itemize environments slightly smaller
% \AtBeginEnvironment{itemize}{\small}

%% Set the left/right column width ratio to 6:4.
\columnratio{0.55}

% Start a 2-column paracol. Both the left and right columns will automatically
% break across pages if things get too long.
\begin{paracol}{2}
\cvsection{Experience}

\cvevent{Lead Senior Firmware Engineer}{\href{https://carbon3d.com}{Carbon, Inc.}}{June 2017 - Current}{Redwood City, California}
\begin{itemize}
\item Led the firmware engineering team at Carbon.
\item Delivered the Automatic Operation (AO) Backpack as a solo firmware engineer.
\item Served as the main firmware engineer on the M3 and M3 Max project—Carbon’s smartest, fastest, and most cost-effective printer to date.
\item Built a Brushless DC Motor Controller, critical to all Carbon products, from the ground up.
\item Developed Hardware-in-the-loop (HIL) testing to identify and fix bugs before shipping.
\item Brought up many new circuit boards and developed an automated HIL system for board bring-up.
\item Established Over-The-Air (OTA) updates for motors and other subsystems in the printer.
\end{itemize}

\cvsection{Languages}
\cvskill{C++}{5}
\cvskill{C}{5}
\cvskill{Python}{4}
\cvskill{Verilog}{2}
\cvskill{Bash}{1}

\divider

\cvsection{Proficiencies}

\cvtag{I2C}
\cvtag{SPI}
\cvtag{UART}
\cvtag{USB}
\cvtag{MQTT}
\cvtag{Protobuf}
\cvtag{Docker}
\cvtag{System Integration}
\cvtag{Board Bring-up}
\cvtag{Linux}
\cvtag{Bazel}
\cvtag{Cmake}
\cvtag{Jenkins}
\cvtag{HIL Testing}
\cvtag{RTOS}
\cvtag{YAML}
\cvtag{Mako}

\cvsection{Education}

\cvevent{B.Sc.\ Mechatronics Engineering}{California State University, Chico}{Aug 2013 - May 2018}{}

\divider

\cvevent{B.Sc.\ Computer Engineering}{California State University, Chico}{Aug 2013 - May 2018}{}


%% Switch to the right column. This will now automatically move to the second
%% page if the content is too long.
\switchcolumn


\divider

\cvsection{Projects}

\cvevent{ \underline{\textbf{\href{https://www.carbon3d.com/products/ao-suite-hardware}{Automation Operation(AO) Backpack}}}}{\href{https://carbon3d.com}{Carbon}}{}{}
\begin{itemize}
\item Worked with Mechanical and Electrical engineers to design behaviors and operate hardware to make printers fully autonomous.
\item Led the effort to move from multiple motor vendors to utilizing a single motor controller (also a project of mine) to reduce costs and increase our ownership of code, as well as enable OTA updateability and logging.
\item Delivered the product in the shortest timeline ever seen at Carbon—on time and under budget as a team of one.
\item Drove firmware development using customer feedback in the cycle and continued to release updates based on that feedback, improving the product over time.
\end{itemize}

\divider

\cvevent{\underline{\textbf{\href{https://www.carbon3d.com/products/m3-3d-printer/}{M3 and M3 Max}}}}{\href{https://carbon3d.com}{Carbon}}{}{}
\begin{itemize}
\item Collaborated with hardware engineers, product managers, and print development engineers to create firmware requirements.
\item Developed all new subsystems in-house to create Carbon’s “smartest” printer.
\item Wrote firmware with extensibility, ease of debugging, and robustness in mind to create a lasting product.
\item Addressed challenges, including supply chain shortages, which necessitated creating new drivers from scratch on a tight timeline.
\end{itemize}

\divider

\cvevent{Brushless DC Motor Controller(BLDC)}{\href{https://carbon3d.com}{Carbon}}{}{}
\begin{itemize}
\item Developed a motor controller from concept to production, featuring cascaded control loops for torque, velocity, relative position, and absolute position control.
\item Created drivers for USB logging, UART API interface, SPI-controlled FET-driver chip, and I2C EEPROM for hardware version control.
\item Designed and implemented hardware fault detection and safety states to prevent damage to mechanics and electronics.
\item Architected Hardware-in-the-loop(HIL) testing using a dynamometer for system and firmware validation.
\end{itemize}

\end{paracol}
\newpage

\cvsection{Other Projects}

\cvevent{Board Testing Rig}{\href{https://carbon3d.com}{Carbon}}{}{}
\begin{itemize}
\item Implemented a method for electrical engineers to hand off new circuit board designs by allowing them to describe hardware changes using YAML
\item YAML files from Electrical Engineers were able to be used to test any new circuit board at a low-level to test for manufacturing correctness
\item Created a bridge so that firmware team could take new YAML file and use it to make changes to code and then test the new code against the file so that there was a single source of truth for the electrical hardware/firmware testing and development.
\end{itemize}
\divider

\cvevent{Microtest}{\href{https://carbon3d.com}{Carbon}}{}{}
\begin{itemize}
\item Developed a firmware image which was able to test core interfaces of the microcontroller
item Server-side application to interface the microcontroller with gtest on Linux, using  \underline{\textbf{\href{https://github.com/nanopb/nanopb}{Nanopb}}}
\item Developed Jenkins-based job to automatically run tests upon pull request.
\item Created ability write firmware that could be automatically called and tested via Jenkins.
\item Recorded statistics like execution time and logs on target processor.
\end{itemize}

\divider

\cvevent{Code Generation Projects}{\href{https://carbon3d.com}{Carbon}}{}{}
\begin{itemize}
\item Worked to develop python-generated firmware to automate different parts of the firmware stack:
	\begin{itemize}
	\item Server <-> embedded system communication.
	\item microcontroller <-> microcontroller communication and commands.
	\item Hardware faults and alerts generated through entire software stack micro -> server -> cloud.
	\end{itemize}
\end{itemize}

\divider

\cvevent{Board Bootloader and Over-The-Air(OTA) Update Software}{\href{https://carbon3d.com}{Carbon}}{}{}
\begin{itemize}
\item Helped build the bootloaders for smart subsystems of printer
\item Extended OTA updates to incorporate updating all smart subsystems.
\item Helped with designs for updater that manages all the firmware versions across the printer/device to only update applicable subsystems.
\end{itemize}

\divider

\end{document}
